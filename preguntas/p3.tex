\item Si la region $1 \leq x \leq 2 $, $0 \leq y \leq 1 $ es denominada $A$, determine el valor de
        \begin{LARGE}
            \begin{equation*}
                \iint_A \frac{1}{x^2 + 2xy + y^2 + 1} dx dy
            \end{equation*}
        \end{LARGE}
\section*{Respuesta}
    \subsubsection*{Introducción}
        Para este problema necesitamos calcular la doble integral de la función dada, para ello calcularemos primero la integral dependiente
        de una variable considerando la otra variable como constante, posteriormente evaluaremos en el espacio determinado como A. Después 
        calcularemos y evaluaremos la segunda integral.
    \subsubsection*{Paso 1: Solución de la integral respecto de x}
        Para resolver la primer integral para $x$ en el espacio dado podemos reescribir la integral de la siguiente forma, esto recordando
        la expansión de un binomio al cuadrado $(a+b)^2 = a^2 + 2ab + b^2$
        \begin{equation*}
            \iint_A \frac{1}{x^2 +2xy + y^2 + 1} = \int \left[ \int_1^2 \frac{1}{(x + y)^2 + 1}dx \right] dy
        \end{equation*}
        Esta integral la podemos simplificar realizando el siguiente cambio de variable:
        \begin{align*}
            u & = x + y \\
            du & = dx \\
        \end{align*}
        si $x$ esta definido en $1 \leq x \leq 2$ entonces tenemos $1 + y \leq x + y \leq 2 + y$ así que $1 + y \leq u \leq 2 + y$.
        \begin{equation*}
            \int \left[ \int_{1+y}^{2+y} \frac{1}{(u)^2 + 1}du \right] dy
        \end{equation*}
        Para la integral respecto de $u$ utilizamos $\int \frac{1}{x^2 + 1} dx = \arctan(x)$
        \begin{align}
            \int \left[ \int_{1+y}^{2+y} \frac{1}{(u)^2 + 1}du \right] dy & = \int \left[ \arctan(u)|_{1+y}^{2+y} \right] dy \\
            & = \int \left[ \arctan(2+y) - \arctan(1+y) \right] dy \\
            & = \int \arctan(2+y) dy - \int \arctan(1+y) dy
            \label{eq:sol3p}
        \end{align}
        Para estas dos integrales resultantes veamos que son similares, para resolverlas ambas al mismo tiempo en una sola notemos 
        que el argumento de las dos es de la forma $(a + y)$, entonces resolvamos
        \begin{equation*}
            \int \arctan(a + y)dy
        \end{equation*}
        Resolvamos esta integral integrando por partes, es decir utilizando la siguiente fórmula de integración:
        \begin{equation*}
            \int udv = uv - \int vdu
        \end{equation*}
        Para nuestro caso sea
        \begin{align*}
            u & = \arctan(a + y) & dv & = dy  \\
            \dfrac{d u}{dy} & = \frac{1}{(a+y)^2 + 1} & \int dv & = \int dy \\
            du & = \frac{1}{(a+y)^2 + 1} dy & v &= y
        \end{align*}
        sustituyendo esto en nuestra fórmula de integración obtenemos
        \begin{equation*}
            \int arctan(a + y) = y \arctan(a + y) - \int \frac{y}{(a + y)^2 +1} dy
        \end{equation*}
        Ahora para resolver esta ultima integral haremos el siguiente cambio de variable $x = (a + y)$, obteniendo asi
        \begin{align*}
            \int \frac{y}{(a + y)^2 + 1} & = \int \frac{x - a}{x^2 + 1} \\
            & = \int \frac{x}{x^2 + 1} - a \int \frac{1}{x^2 + 1} \\
            & = \int \frac{x}{x^2 + 1} - a \arctan(x)
        \end{align*}
        Para la primer integral que falta realizamos el siguiente cambio de variable $ s = x^2 + 1$ asi que $ds = 2x dx $,
        utilizaremos también $ \int \frac{1}{v}dv = \ln(v)$
        \begin{align*}
            \int \frac{x}{x^2 + 1} dx & = \int \frac{x}{s}\frac{2}{2x}ds \\
            & = \frac{1}{2} \int \frac{1}{s}ds \\
            & = \frac{1}{2} \ln(s) \\
            & = \frac{\ln(x^2 +1)}{2}
        \end{align*}
        Sustituyendo este resultado y el anterior y regresando a la variable $y$, obtenemos
        \begin{align*}
            \int arctan(a + y) & = y \arctan(a + y) - \left[ - \arctan(a + y) + \frac{\ln((a + y)^2 +1)}{2} \right] \\
            & =  y \arctan(a + y) + a \arctan(a + y) - \frac{\ln((a + y)^2 +1)}{2}
        \end{align*}
        Utilizamos este ultimo resultado en \eqref{eq:sol3p} con los valores respectivos de $a$ para obtener el resultado final
        \begin{align*}
            \int_0^1 \left[ \int_{1+y}^{2+y} \frac{1}{(u)^2 + 1}du \right] dy & = \int_0^1 \arctan(2+y) dy - \int_0^1 \arctan(1+y) dy \\
            & = \left[y \arctan(2 + y) + 2 \arctan(2 + y) - \frac{\ln((2 + y)^2 +1)}{2}\right]|_0^1 \\
            & \hspace{1cm} - \left[y \arctan(1 + y) + \arctan(1 + y) - \frac{\ln((1 + y)^2 +1)}{2}\right]|_0^1 \\
            & = \left[(y + 2)(\arctan(2 +y)) - \frac{\ln((2 + y)^2 +1)}{2})\right]|_0^1 \\
            & \hspace{1cm} -  \left[(y + 1)(\arctan(1 +y)) - \frac{\ln((1 + y)^2 +1)}{2})\right]|_0^1 \\
            & = (3)(\arctan(3)) - \frac{\ln(10)}{2}) - (2)(\arctan(2)) - \frac{\ln(5)}{2}) \\
            & \hspace{1cm} -\left[(2)(\arctan(2)) - \frac{\ln((5)}{2}) - \arctan(1) - \frac{\ln(2)}{2})\right]
        \end{align*}
    \subsection*{Conclusión}
        Calculando los valores numéricos obtenemos
        \begin{equation*}
            \boxed{ \int_0^1 \left[ \int_{1+y}^{2+y} \frac{1}{(u)^2 + 1}du \right] dy = 0.215512 }
        \end{equation*}