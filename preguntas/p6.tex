\item Usando una expansión en serie, determine el valor de la siguiente integral con un error menor a $10^-8$.
        \begin{LARGE}
            \begin{equation*}
                \int_0^{0.1} \frac{dx}{\sqrt{x^3 + 1}}
            \end{equation*}
        \end{LARGE}
\section*{Respuesta}
    \subsubsection*{Introducción}
        Para este problema necesitamos primero expandir la función a integrar como una serie de potencias, para despues calcular la integral
        de esa serie, cortando la serie hasta obtener un valor de error menor al requerido. Para ello utilizaremos la expansion en serie de
        Maclaurin.
    \subsubsection*{Paso 1: Expansión en serie de la función}
        Si $f$ tiene n derivadas en $x = 0$ el n-ésimo polinomio de Maclaurin de la función $f$ se define como:
        \begin{equation*}
            p_n(x) = f(0) + f'(0)x + \frac{f''(0)}{2!}x^2 + ... +\frac{f^{(n)}(0)}{n!}x^n
        \end{equation*}
        Donde $f^{(n)}$ es la derivada de orden $n$ de la función $f$, $n! = n(n-1)(n-2)...(2)(1)$ es el factorial de $n$.
        Como solo nos interesa la integral con un error menor a $10^8$ y como nuestro límite superior es $0.1$ y $10^8 = (0.1)^{10} $, bastará
        con que nuestro polinomio sea de un grado menor a 10.

        Calculemos entonces las derivadas
        \begin{align*}
            f'(0) & = \dfrac{d}{dx}(\frac{1}{\sqrt{x^3 + 1}})\hspace{0.5 cm} \text{haciendo el cambio de variable} \hspace{0.1 cm} u = x^3 + 1 \\
            & = \dfrac{d}{du}\frac{1}{\sqrt{u}} \dfrac{d}{dx}(x^3 + 1) \\
            & = 3x^2(\frac{-1}{2u^{3/2}}) \\
            & = 3x^2(\frac{-1}{2(x^3 + 1)^{3/2}}) \\
            & = - \frac{3x^2}{2(x^3 + 1)^{3/2}} \hspace{0.5 cm} \text{evaluando en} \hspace{0.1 cm} x = 0 \\
            & = 0
        \end{align*}
        Para las derivadas posteriores utilizaremos el mismo cambio de variable en pasos intermedios, además de la regla del producto de una derivada
        $\dfrac{d}{dx}(f(x)\cdot g(x)) = f(x) \dfrac{d g(x)}{dx} + g(x)\dfrac{d f(x)}{dx}$
        \begin{align*}
            f''(0) & = \dfrac{d}{dx}(- \frac{3x^2}{2(x^3 + 1)^{3/2}}) \\
            & = -\frac{3}{2} \dfrac{d}{dx}((x^2)((x^3 + 1)^{-3/2})) \\
            & = -\frac{3}{2} \left[2x((x^3 + 1)^{-3/2})) + x^2(3x^2(\frac{-3}{2}(x^3 + 1)^{-5/2}))\right] \\
            & = \frac{3}{4}\left[x(5x^4 - 4)(x^3 + 1)^{-5/2}\right] \\
            & = 0
        \end{align*}
        Podemos continuar con los cálculos de las derivadas, pero para hacerlo mas practico podemos desarrollar un script de Python que nos ayude
        a calcularlo
        \lstinputlisting{codes/macPol.py}
        Con la ayuda de este script encontramos que la serie con la aproximación buscada es:
        \begin{equation}
            \frac{1}{\sqrt{x^3 + 1}} =   1 - \frac{1}{2}x^3 + \frac{3}{8}x^6 - \frac{5}{16}x^9
        \end{equation}
        Sustituyendo esto en nuestra integral inicial obtenemos
        \begin{align*}
            \int_0^{0.1} \frac{dx}{\sqrt{x^3 + 1}} & =  \int_0^{0.1} \left(1 - \frac{1}{2}x^3 + \frac{3}{8}x^6 - \frac{5}{16}x^9\right)  dx\\
            & =   \int_0^{0.1} 1 dx -  \frac{1}{2} \int_0^{0.1} x^3 dx + \frac{3}{8} \int_0^{0.1} x^6 dx - \frac{5}{16} \int_0^{0.1} x^9 dx \\
        \end{align*}
        Para resolver estas integrales utilizamos $ \int x^n = \frac{x^{n+1}}{n+1} $
        \begin{align}
            \int_0^{0.1} \frac{dx}{\sqrt{x^3 + 1}} & = \int_0^{0.1} 1 dx -  \frac{1}{2} \int_0^{0.1} x^3 dx + \frac{3}{8} \int_0^{0.1} x^6 dx 
            - \frac{5}{16} \int_0^{0.1} x^9 dx \\
            & = \left[x -  \frac{1}{2} \frac{x^4}{4} + \frac{3}{8} \frac{x^7}{7} -  \frac{5}{16}  \frac{x^{10}}{10} \right]_0^{0.1} \\
            & = (0.1) -  \frac{1}{2} \frac{(0.1)4}{4} + \frac{3}{8} \frac{(0.1)^7}{7} -  \frac{5}{16}  \frac{(0.1)^{10}}{10}
        \label{eq:sol6p}
        \end{align}

\subsection*{Conclusión}
    La solución de esta integral es:
    \begin{equation*}
        \boxed{\int_0^{0.1} \frac{dx}{\sqrt{x^3 + 1}} = 0.099987562496875}
    \end{equation*}