\item Utilice las condiciones iniciales dadas para determinar la solución particular del siguiente sistema:
    \begin{LARGE}
        \begin{equation*}
            \ddot{x} + 6\dot{x} + 9x = 3e^{-3t}
        \end{equation*}
    \end{LARGE}
    Las condiciones iniciales son $x(0)=0$ y $\dot{x}(0)=1$.
\section*{Respuesta}
    \subsubsection*{Introducción}
        Para este problema necesitamos calcular la solución del sistema dadas las condiciones iniciales, para
        ello dado que se trata de una ecuación diferencial ordinaria (ODE) no homogénea, su solución se compone de 
        la suma de una solución complementaria y una solución particular, calculáremos primero la parte complementaria que es
        la solución de la ecuación homogénea asociada, posteriormente calcularemos la solución particular y finalmente
        utilizaremos las condiciones iniciales para hallar los valores de las constantes que acompañan la solución.
    \subsubsection*{Paso 1: Solución de la ecuación homogénea asociada}
        La ecuación homogénea asociada a nuestra ecuación es:
        \begin{equation}
            \ddot{x} + 6\dot{x} + 9x = 0
            \label{eq:diff}
        \end{equation}
        Para este tipo de ecuaciones sus soluciones tienen la siguiente forma: $x = e^{\lambda t} $, donde debemos encontrar el
        valor de $\lambda$ sustituyendo $ x $ en \eqref{eq:diff}. Obtenemos asi utilizando 
        $\dfrac {d( e^{\lambda t} )}{d t}=\lambda e^{\lambda t}$:
        \begin{equation*}
            \begin{split}
                \ddot{x} + 6\dot{x} + 9x & = \lambda^2 e^{\lambda t} + 6\lambda e^{\lambda t} 9 e^{\lambda t}\\
                & = e^{\lambda t}(\lambda^2 + 6\lambda + 9) \\
                & = 0
            \end{split}
        \end{equation*}
        Dado que $ e^{\lambda t} $ nunca es cero para valores finitos de t, bastara con resolver:
        \begin{equation*}
            \lambda^2 + 6\lambda + 9 = 0
        \end{equation*}
        para encontrar los valores de $\lambda$, podemos entonces agrupar de la siguiente forma
        \begin{equation*}
            \begin{split}
                \lambda^2 + 6\lambda + 9 & = (\lambda + 3)(\lambda + 3)\\
                & = (\lambda + 3)^2 \\
                & = 0
            \end{split}
        \end{equation*}
        las soluciones de esta ecuación nos devuelven los valores de $\lambda_1 = -3$ y $\lambda_2 = -3$, dado que los valores son múltiplos
        enteros (particularmente el mismo valor), la solución de la ecuación homogénea sera $ c_1 e^{\lambda_1 t} + c_2 e^{\lambda_1 t} t$
        donde $c_1$ y $c_2$ son valores enteros que calcularemos mas adelante. Sustituyendo los valores obtenemos asi la solución 
        complementaria
        \begin{equation}
            x_c = c_1 e^{-3 t} + c_2 e^{-3 t} t
            \label{eq:xc}
        \end{equation}
    \subsubsection*{Paso 2: Solución particular por el método de coeficientes indeterminados}
        Dado que la ecuación homogénea asociada nos devolvió una solución de la forma $t e^{-3 t}$ la solución particular debe ser de
        la forma $x_p = a e^{-3 t} t^2 $.

        Calculemos primero $\dfrac{d^2(x_p)}{dt^2}$ y $\dfrac{d(x_p)}{dt}$
        \begin{equation}
            \begin{split}
                \dfrac{d(x_p)}{dx} & = \dfrac{d(a e^{-3 t} t^2)}{dt}\\
                & = a \dfrac{d(e^{-3 t} t^2)}{dt} \\
                & = a\left[\dfrac{d(e^{-3 t})}{dt}t^2 + e^{-3t}\dfrac{d(t^2)}{dt}\right] \\
                & = a(-3e^{-3t}t^2 + 2e^{-3t}t)
            \end{split}
            \label{eq:p1}
        \end{equation}
        \begin{equation}
            \begin{split}
                \dfrac{d^2(x_p)}{dt^2} & = \dfrac{d(\dfrac{d(x_p)}{dt})}{dt} \\
                & = \dfrac{d(a[-3e^{-3t}t^2 + 2e^{-3t}t])}{dt} \\
                & = a\left[-3\dfrac{d(e^{-3t}t^2)}{dt} + 2\dfrac{d(e^{-3t}t)}{dt}\right] \\
                & = a\left[-3(\dfrac{d(e^{-3 t})}{dt}t^2 + e^{-3t}\dfrac{d(t^2)}{dt}) + 2(\dfrac{d(e^{-3t})}{dt}t+e^{-3t}\dfrac{d(t)}{dt})\right] \\
                & = a\left[-3(-3e^{-3t}t^2 + 2e^{-3t}t)+ 2(-3e^{-3t}+e^{-3t})\right] \\
                & = a(9e^{-3t}t^2 - 12e^{-3t}t + 2e^{-3t})
            \end{split}
            \label{eq:p2}
        \end{equation}
        sustituyendo \eqref{eq:p1} y \eqref{eq:p2} en nuestra ecuación inicial obtenemos el valor de $a$
        \begin{equation*}
            \begin{split}
                \ddot{x} + 6\dot{x} + 9x & = 3e^{-3t} \\
                a(9e^{-3t}t^2 - 12e^{-3t}t + 2e^{-3t}) + 6a(-3e^{-3t}t^2 + 2e^{-3t}t) + 9e^{-3t}t^2 & = 3e^{-3t} \\
                a(9t^2 - 12t + 2 -18t^2 + 12t + 9t^2)e^{-3t} & = 3e^{-3t} \\
                2a & = 3 \\
                a & = \frac{3}{2}
            \end{split}
        \end{equation*}
        De modo que nuestra solución particular es 
        \begin{equation}
            x_p = \frac{3}{2}e^{-3t}t^2
            \label{eq:xp}
        \end{equation}
        la suma de nuestra solución complementaria \eqref{eq:xc} y nuestra solución particular\eqref{eq:xp} es nuestra solución completa.
        \begin{equation}
            \begin{split}
                x & = x_c + x_p \\ 
                & = c_1 e^{-3 t} + c_2 e^{-3 t} t + \frac{3}{2}e^{-3t}t^2 \\
                & = (c_1 + c_2 t + \frac{3}{2} t^2)e^{-3t}
            \end{split}
            \label{eq:sol1p}
        \end{equation}
    \subsubsection*{Paso 3: Calcular los valores de los coeficientes}
        Para calcular los valores de los coeficientes $c_1$ y $c_2$ utilizaremos las condiciones iniciales, para ello utilizando
        \eqref{eq:sol1p} y recordando que $e^0 = 1$ calculemos
        \begin{equation}
            \begin{split}
                x(0) & = 0 \\
                (c_1 + c_2 * (0) + \frac{3}{2}(0)^2)e^{-3*0} & = 0\\
                c_1 & = 0
            \end{split}
            \label{eq:c1}
        \end{equation}
        \begin{equation}
            \begin{split}
                \dot{x}(0) & = 0 \\
                \dfrac{d([c_1 + c_2 t + \frac{3}{2} t^2]e^{-3t})}{dt}|_{t=0} & = 1\\
                \dfrac{d(c_1 + c_2 t + \frac{3}{2}t^2)}{dt}e^{-3t}|_{t=0} + [c_1 + c_2 t + \frac{3}{2}]\dfrac{d(e^{-3t})}{dt}|_{t=0} & = 1 \\
                (c_2 + 3t)e^{-3t}|_{t=0} + [c_1 + c_2 t + \frac{3}{2}t^2](-3e^{-3t})|_{t=0} & = 1 \\
                (c_2 + 3*(0))e^{-3*(0)} + [c_1 + c_2*(0) + \frac{3}{2}(0)^2](-3e^{-3*(0)}) & = 1 \\
                c_2 - 3c_1 +  & = 1\\
                c_2 - 3*(0) & = 1\\
                c_2 & = 1
            \end{split}
            \label{eq:c2}
        \end{equation}
    \subsubsection*{Conclusión}
        Sustituyendo los valores de $c_1$ y $c_2$ obtenidos en \eqref{eq:c1} y \eqref{eq:c2} respectivamente en \eqref{eq:sol1p}
        obtenemos la solución final.
        \begin{equation*}
            \boxed{x =  \left(1 + \frac{3}{2} t^2\right)e^{-3t}}
        \end{equation*}